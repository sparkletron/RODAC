\documentclass{article}
\usepackage{graphicx}
\usepackage[hidelinks]{hyperref}
\begin{document}
  \begin{titlepage}
    \begin{center}

    {\Huge RODAC}

    \vspace{25mm}

    \includegraphics[width=0.90\textwidth,height=\textheight,keepaspectratio]{src/img/SPARKLETRON.png}

    \vspace{25mm}

    \today

    \vspace{15mm}

    {\Large Jay Convertino}

    \end{center}
  \end{titlepage}

  \tableofcontents

  \newpage

  \section{Usage}

  \subsection{Introduction}

  \par
  This manual describes how to use the RODAC (Retro Only Device Application Creation) system for development.
  Items such as how to build included apps, what the structure of the system looks like, and how to create your
  own app is included. The final section are links to doxygen generatated documentation about the drivers used
  by this system.

  \subsection{Dependecies}

  \par
  COMING SOON(TM)

  \subsection{Building}

  \par
  COMING SOON(TM)

  \subsubsection{hello\_world}

  \par
  COMING SOON(TM)

  \subsubsection{multicart}

  \par
  COMING SOON(TM)

  \subsection{Directory Guide}

  \par
  Below highlights important folders from the root of RODAC.

  \begin{enumerate}
    \item \textbf{docs} Contains all documentation related to this project.
      \begin{itemize}
        \item \textbf{arch} Contains all architecture docs related to retro systems.
        \item \textbf{manual} Contains user manual and wiki that are generated from the same latex source.
      \end{itemize}
    \item \textbf{apps} Contains source code in C for the applications to run on the target architecture.
      \begin{itemize}
        \item \textbf{hello\_world} Example hello world application. Targets all architectures.
        \item \textbf{mutlicart} Example multicart application, written for the coleco only.
      \end{itemize}
    \item \textbf{drivers} Contains all source code related to the project.
      \begin{itemize}
        \item \textbf{gisnd} Simple driver for the GI AY-3-8910 sound chip and its variants.
        \item \textbf{sn76489} driver for the TI SN76489 sound chip.
        \item \textbf{tms99XX} driver for all TMS99XX and TMS9XXX video chips.
      \end{itemize}
  \end{enumerate}

  \newpage

  \section{Application Creation}

  \newpage

  \section{System Creation}

  \newpage

  \section{Architecture}
  \subsection{General Desciption}

  \par
  system by system

  \subsection{System Targets}

  \newpage

  \subsection{Application Targets}
  \subs

  \section{Driver Documentation}

  \subsection{TMS99XX Doxygen}
  \href{https://sparkletron.github.io/RODAC/manual/dox/tms99XX/html/index.html}{TMS99XX HTML Doxygen}

  \subsection{SN76489 Doxygen}
  \href{https://sparkletron.github.io/RODAC/manual/dox/sn76489/html/index.html}{SN76489 HTML Doxygen}

  \subsection{GISND Doxygen}
  \href{https://sparkletron.github.io/RODAC/manual/dox/gisnd/html/index.html}{GISND HTML Doxygen}

\end{document}
